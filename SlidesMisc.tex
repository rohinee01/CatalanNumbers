\documentclass[handout, dvipsnames]{beamer}
\mode<presentation>{}
\usepackage[utf8]{inputenc}
\usepackage{amsmath, amssymb, amsfonts, amsthm, mathtools, mathrsfs}
\setbeamertemplate{theorems}[numbered]
\title{Catalan Numbers}
\author{Maths and Physics Club}
\date[23-08-2020]{23rd August 2020}
\institute[IITB]{IIT Bombay}
\usetheme{Warsaw}
% \usecolortheme{beetle}
\usepackage{graphicx}

\newcommand{\id}{\operatorname{id}}
\renewcommand{\exp}{\operatorname{exp}}

\theoremstyle{definition}
\newtheorem{defn}{Definition}
\newtheorem{prop}{Proposition}
\newtheorem{thm}{Theorem}

%Slide 1
\begin{document}
\begin{frame}
    \titlepage
\end{frame}

%slide 2
\begin{frame}{Introduction}
    %yet to decide
\end{frame}

%slide 3
\begin{frame}{Polygon Triangulation}
For n$\geq$3, let $C_{n-2}^1$ denotes the number of ways to \emph{triangulate} a n-sided convex polygon. Find $C_{n-2}^1$ for n$\geq$3. 
\begin{figure}
    \centering
    \includegraphics[width=7.5 cm]{PolyTriang.png}
    \caption{Illustration for small values of n}
    \label{fig:PolyTriang}
\end{figure}
\end{frame}

%slide 4
\begin{frame}{Approach- Recursion!!!}
    \begin{enumerate}
        \uncover<2->{\item Define $C_0^1=$1 for n$=$2. (Why?)}
        \uncover<3->{\item For a given polygon of n$\geq$3 vertices, Fix an edge, and consider the triangle containing that edge. ($n-2$ ways)
        \begin{figure}
            \centering
            \includegraphics[width=4cm]{PolyTri2.JPG}
            \caption{Illustration for $n=8$}
            \label{fig:PolyTri2}
        \end{figure}}
        \uncover<4->{\item Triangulate \emph{left} k gon in $C_{k-2}^1$ ways, and \emph{right} $n-k+1$ gon in $C_{n-k-1}^1$ ways.}
        \uncover<5->{\item Combine! ($C_{k-2}^1 C_{n-k-1}^1$ ways)}
        \uncover<6->{\item Sum it over $n-2$ expressions}
    \end{enumerate}
\end{frame}

%slide 5
\begin{frame}{Polygon Triangulation-Recursion}
    Define $C_0^1=$1.\\
    For n$\geq$3, let $C_{n-2}^1$ denotes the number of ways to triangulate a n-sided convex polygon.\\\\
    Then the sequence $\{C_n^1\}_{n=0}^\infty$ is given by,  
    $$C_{n-2}^1 = \sum_{k = 0}^{n-1}\ C_{k-2}^1 C_{n-k-1}^1$$. 
\end{frame}
<return>

%slide 6
\begin{frame}{Hands across a table}
    For n$\geq$1, suppose $2n$ people are sitting around a round table. Let $C_n^2$ denotes the number of ways in which all of them be can simultaneously shake hands with another person at the table in such a way that none of the arms cross each other. Find $C_n^2$.
    \begin{figure}
        \centering
        \includegraphics[width=7.5cm]{HandCross.JPG}
        \caption{Illustration for small values of n}
        \label{fig:my_label}
    \end{figure}
\end{frame}

%slide 7
\begin{frame}{Approach- again recursion!}
    \begin{enumerate}
        \uncover<2->{\item Again define $C_0^2=$1 for n$=$0. (Why?)}
        \uncover<3->{\item For a given n$\geq$1, Choose a pair. Legal pattern: even number of pairs on each side.}
        \uncover<4->{\item Again recursion!\\
        k pairs on the left \longrightarrow $C_{k}^2$ ways.\\ 
        $n-k+1$ pairs on right \longrightarrow $C_{n-k-1}^2$ ways.}
        \uncover<5->{\item Combine! ($C_{k}^2 C_{n-k-1}^2$ ways)}
        \uncover<6->{\item Sum it over all $n$ expressions.}
        \uncover<7->{\item Recurrence relation $$C_{n}^2 = \sum_{k = 0}^{n-1}\ C_{k}^2 C_{n-k-1}^2$$.}
        \uncover<8->{\item Analogy with $\{C_n^1\}_{n=0}^\infty$ ??!}
    \end{enumerate}
\end{frame}

%Sangraam Slides

%slide 8
\begin{frame}{Parenthesis}
    For n$\geq$1, we have n pairs of parentheses and we would like to form valid groupings of them. “valid” means that each open parenthesis has a matching closed parenthesis.\\
    Define a sequence as $C_0^6=$1, and $C_n^6$ as the number of such valid groupings, for n$\geq$1. Find the sequence. 
    \begin{figure}
        \centering
        \includegraphics[width=9.5cm]{Paranthesis.JPG}
        \caption{Illustration for small values of n}
        \label{fig:my_label}
    \end{figure}
\end{frame}

%slide 9
\begin{frame}{Approach}
    
\end{frame}

%slide 10
\begin{frame}{Mountain ranges}
    How many “mountain ranges” can you form with n upstrokes and n downstrokes that all stay above the original line? Define $C_0^7=1$ and find the sequence as the solution to this problem as $C_n^7$, for n$\geq$1. 
    \begin{figure}
        \centering
        \includegraphics[width=7.5cm]{MountRange.JPG}
        \caption{Illustration for small values of n}
        \label{fig:my_label}
    \end{figure}
\end{frame}

%slide 11
\begin{frame}{Approach}
    Analogy to previously solved problem?\\
    left parenthesis \longrightarrow upstroke \\
    right parenthesis \longrightarrow downstroke \\
    Both arrangements have the same condition- while going from left to right, number of left parenthesis (or upstrokes) is not less than the number of right parenthesis (or downstrokes).\\\\
    So both the problems have exactly same solution set. 
\end{frame}

%slide 12
\begin{frame}{Diagonal avoiding paths}
    In a grid of n×n squares, how many paths are there of length 2n that lead from the upper left corner to the lower right corner that stay on or above the main diagonal?
    \begin{figure}
        \centering
        \includegraphics[width=7cm]{DiagProb.JPG}
        \caption{An illustration }
        \label{fig:my_label}
    \end{figure}
\end{frame}

%slide 13
\begin{frame}{Approach}
    Again, an analogy to previously solved problem?\\
    \begin{figure}
        \centering
        \includegraphics[width=9cm]{Analogy1.JPG}
        \caption{Illustrating the analogy}
        \label{fig:my_label}
    \end{figure}\\
    For a given n, bijection between number of mountain ranges and diagonal avoiding paths??\\\\
    So both the problems have exactly same solution set. 
\end{frame}

%slide 14
\begin{frame}{Election problem}
    Two candidates A and B are running in an election. There are 2n voters who vote sequentially for one of the two candidates. At the end, each candidate receives n votes. What is the probability that A never trails B in the voting?\\\\
    Again, an identical problem. (Why?)\\\\
    Define a \emph{ballot sequence} of length $2n$, as a sequence with $n$ 1's and $n -1's$ such that every partial sum is non-negative. \\\\
    All we have to do is find the number of ballot sequences. 
\end{frame}

%slide15
\begin{frame}{A chess puzzle}
    A \emph{helpmate} is a type of chess problem where white and black \emph{cooperate} in order to achieve the goal of checkmating black.\\
    A \emph{serieshelpmate in 34} is a chess problem in which black makes 34 consecutive moves, after which white checkmates black in one move. \\\\
    \emph{Queue Problems}- 15 problems of similar nature.
\end{frame}

%slide16
\begin{frame}{A chess puzzle}
    \begin{figure}
        \centering
        \includegraphics[width=7cm]{Screenshot_20200817-181011.jpg}
        \caption{Consider the following chess position}
        \label{fig:my_label}
    \end{figure}
    \end{frame}
    
%slide17
\begin{frame}{A chess puzzle}
     \uncover<1->{For a serieshelpmate in 34, how many different solutions are there?}\\
    \uncover<2->{The answer is $C_17$. (Why?)}\\
    \uncover<3->{There was a chess puzzle constructed by Stanley with $C_9$ solutions.}
\end{frame}
   
%slide18
\begin{frame}{Generalizations}
    \begin{itemize}
        \uncover<1->{\item \emph{Super Catalan Numbers} are defined as $\frac{(2m)!\cdot(2n)!}{(m+n)!n!m!}$.}
        \uncover<2->{\item For $m=1$, this is just two times the ordinary Catalan numbers. For $m=n$, the numbers have an easy combinatorial description. However, other combinatorial descriptions are only known for $m=2$ and $m=3$. It is an open problem to find a general combinatorial interpretation.}
        \uncover<3->{\item \emph{Fuss Catalan Numbers, Narayana numbers, Motzkin numbers, Schr¨oder numbers} (Ugghh what's thatt://)}
    \end{itemize}
\end{frame}

%slide19
\begin{frame}{Catalan Numbers}
    \begin{figure}
        \centering
        \includegraphics{graph.JPG}
        \caption{Representation of Catalan Numbers}
        \label{fig:my_label}
    \end{figure}
\end{frame}

%slide20
\begin{frame}{Catalan Numbers}
    %Collage- visual forms of CNs. Thank you.
\end{frame}
\end{document}
